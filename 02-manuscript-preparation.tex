\section{MANUSCRIPT PREPARATION}
According to the call for papers, papers must report original, previously unpublished findings in the fields of sustainable development of energy, transport, water, food and environment systems, their integration, as well as technical, environmental, economic and social perspectives.

The papers should be written using Microsoft Word, or if written in any other application, it should be ported by authors to Microsoft Word format. Authors that find this unacceptable should contact the technical team by e-mail: sdewes2026@sdewes.org.

The text should be single-spaced. If superscripts or subscripts make this a problem, wider spacing may be necessary. Leave double spaces between paragraphs. Begin paragraphs flush at the left margin without indentation. The typing area of all pages should be 160 x 247 mm, whichever size of paper is used, with equal margins on left and right. If you preferably use A4 all the margins should be set to 25 mm. Each page should be completely filled with typing and/or diagrams (except perhaps the last page). The number of pages is not limited, but have in mind, for easier preparation for special issue publishing (applicable to archival papers), it would be wise to keep it in the range from eight to twelve.
\\
% In Latex: A free line inidcates a paragraph. No need for an extra \\. It is used here to realize the format of the Word template.

References should be numbered consecutively in the order they are mentioned \cite{ritchie1983}, using Arabic numerals \cite{kincaid1991, erdas1983} in square brackets \cite{fauchais1994}. Any use of lumped references as in \cite{ritchie1983, kincaid1991, sdewes2019} should be avoided to provide sufficient representation of the main contribution of each referenced paper.
\\

Pages numbering should start from page 1. 
\\

The paper should be cited as:

Authors, Paper title, Proceedings of the 21st Conference on Sustainable Development of Energy, Water, and Environment Systems, SDEWES2026.nnnn, 1-m

where nnnn is the submission code and m is the last page of the paper. 
\\

Authors should use Times or Times New Roman with a 12-point character size for the text. The text should be left and right justified.

\section{LAYOUT}
The layout of the paper should follow the style of this document, starting with a title, name(s) of author(s) and affiliation(s). 

\subsection{Title}
The title should appear 32 mm below the top edge of the page. It should be brief, clear and descriptive. Use Times New Roman 14 bold letters, centred on the width of the typing area. Please use proper title format, not all upper-case.

\subsection{Affiliations}
Authors’ names, affiliations, and e-mails should be Times New Roman 12 but not bold. Leave one blank line between the title and the affiliation(s).

\subsection{Abstract}
A \textbf{brief} abstract (50 - 150 words) should appear beneath the affiliation of the author(s). It should give an account of the most relevant contributions of the paper. It is also important to indicate briefly the goal, the methods, the results, and conclusion. Avoid abbreviations, diagrams, and references. It must be complete and understandable without reference to the text. Leave two blank lines between the Author’s affiliation and the Abstract. Type the word ABSTRACT in capital letters as in the first page in Heading 2 style. In addition, during manuscript submission, ensure that any possible revision in the accepted abstract is also updated in the Comet system.

\section{HEADINGS}
Please use the format adopted here to divide your paper into \textbf{sections} and \textbf{subsections} as applicable, in which first-level headings (Heading 2 style) are in bold capitals, left aligned, Times New Roman 12, with 12 pt spacing before and 6 pt spacing after. It is expected that research papers are coherently organized into sections for the Introduction, Method, Results and Discussion as well as the Conclusion. Ensure the use of “Methods” or “Materials and Methods” rather than the use of “Methodology,” which is the study/analysis of methods to be used only when addressing epistemologies/ontologies. As a whole, the research paper should progress from an introduction, including a literature review, to a sufficiently detailed description of the method to prove or disapprove the hypothesis. This should be followed by a systematic presentation of the results with discussion and then the conclusion. The conclusion should elaborate on the relevance of the results for addressing any identified gap in the literature. 

\subsection{Second level headings}
Second level headings should be in bold lower case (initial capital), left aligned, Heading 3 style, Times New Roman 12, with 12 pt spacing before and 3 pt spacing after. The inclusion of paragraphs between main section headings and any second level headings will be beneficial to provide an overview of the content of the entire section prior to proceeding to additional detail.  

\subsubsection{Third-level headings.} Third-level headings should be placed at the beginning of a paragraph.  Capitalize only first letter of the whole subhead and underline it (if possible, make the subhead italic); follow it by a period and two letter spaces; then begin typing the text on the same line and continue the text without indenting again. Leave one line space above.

\subsubsection{Equations, units, symbols, etc.} Equations should be typed neatly in position with appropriate space above and below to distinguish them from the text. Equations should be either centred or placed flush left, and assigned a number that should appear in parentheses flush to the right margin.

\begin{equation}
    a=\frac{b^\alpha_i}{c}
\end{equation}

Subscripts and superscripts should clearly be typed as such, and the manuscript should be reviewed carefully to ensure there is no ambiguity in presentation. Numbers and letters that are intended to be subscripts or superscripts should not align with the rest of the text.
\\

Do not use punctuation at ends of equations. Greek letters and other symbols should be typed. All data should be reported in SI units. Decimals should always be shown by periods and not by commas or centred dots.
\\

Examples of correct form of data and units are 20\%, \SI{15}{\celsius}, \SI{30}{ml/min}, $NPV = \SI{5}{EUR}$, \SI{3.9}{kEUR/t}, $\geq 1000$ etc. 
\\

All equation variables should be identified within the manuscript text in addition to their inclusion in the Nomenclature at the end of the manuscript to facilitate clarity of presentation.

\subsubsection{Figures.}  Care should be taken to ensure that figures are contained within the typing area. All original drawings should be prepared, if possible, for a uniform scale of reduction. As a general rule, lettering in the figures should be comparable to that in the text.
\\

Figures should be numbered consecutively, e.g. \cref{fig:example} or \Cref{fig:example}, with a single letter space between the word “Figure” and the Arabic numeral. Place figures centred on the width of the text page and either at the top or bottom of the page as close as possible to their first mention in the text. Centred one line below the illustration, type the word “Figure” (in upper and lower case) and its number followed by a period and two-letter space. Then type the legend single-spaced, with an initial capital for the first word and for proper nouns only. Example:

\begin{figure}[h!]
    \centering
    \includegraphics[scale=1]{figures/example.pdf}
    \caption{A figure}
    \label{fig:example}
\end{figure}

Each illustration should have at least a one-line space above the illustration, a one-line space between the illustration and the legend, and at least a one-line space between the legend and the start of the text. All illustrations should be pasted in the file. Appropriate space should be left above and below to the figure legend to ensure that the legend does not become confused with the text.

\subsubsection{Tables.} Table captions should appear above the respective table. Each table should have at least a one-line space both above the table and between the table and the start of the following text.\\

When tables are mentioned in the text, they should be referred to as \Cref{tab:comparison}, \Cref{tab:comparison}, i.e., with a single letter space between the word “Table” and the Arabic numeral.\\

The word “Table” should be capitalized, single-spaced and centred with the table number, followed by a period, two spaces and the table caption, above the table. Use horizontal rules above and below to separate title from column heads, ranks within column heads, column heads from table body, and table body from table footnotes or source. For example:

\begin{table}[h!]
    \centering
    \caption{Comparison between theory and experiment}
    \label{tab:comparison}
    \begin{tabular}{ccccc}
         \toprule
         Date of test & \multicolumn{2}{c}{Theoretical value} & \multicolumn{2}{c}{Experimental value} \\
         & \multicolumn{2}{c}{(cm)} & \multicolumn{2}{c}{(cm)} \\
         \midrule
         & Left & Right & Left & Right \\
         \midrule
         January 1 & 17.45 & 3.81 & 16.98 & 3.99 \\
         March 3 & 21.43 & 6.45 & 22.56 & 6.91 \\
         \bottomrule
    \end{tabular}
\end{table}

Authors should ensure that a table does not flow from one page to the next page. Tables should occupy only as much space as is required. 

\subsubsection{Footnotes.} Since footnotes tend to interrupt the natural flow of ideas in manuscript, they should be limited in number and used to indicate (a) acknowledgement of funding or sponsorship, or (b) copyright information or credit line if the material has been published previously. Footnotes should be numbered or identified by symbols: * † ‡. The footnote\footnote{This is the format of a footnote.} should be separated from the text by a one-line space and a 5 cm overbar. Start each footnote on a separate line at the left margin, typing the superscript symbol at the margin and immediately beginning the text of the footnote. Use the Times New Roman 10 font.

\subsubsection{Acronyms/abbreviations/chemical formula.} Any acronym/abbreviation and/or chemical formula should be introduced when used for the first time in the main manuscript text. The order should be the full name followed by the introduced form in parentheses, such as Sustainable Development of Energy, Water and Environment Systems (SDEWES) and carbon dioxide (\ce{CO2}). All introduced forms of acronyms, abbreviations and chemical formula should be used consistently thereafter, including the conclusion. Acronyms, abbreviations or chemical formula should not be used in the title, abstract and section/subsection headings.

\subsubsection{References.} Reference formatting should conform to the examples in the reference list. Reference management software can be used for the references, which will be of assistance for reference formatting also during any journal submission if the manuscript would be invited. 

\section{NON-ENGLISH SPEAKING AUTHORSFINALIZATION OF THE MANUSCRIPT}
All authors are requested to conduct a thorough proof-reading of the text to uphold language and organizational quality. The use of first person singular or plural should be avoided in research papers. Authors from non-English speaking countries are requested to find persons who are competent in English and familiar with the scientific language who can edit their manuscripts before submission. Reviewers or Editors must not be relied upon to make corrections of English expression, spelling, etc. As there is no copy editing stage for camera-ready manuscripts, it is the responsibility of authors to ensure that the presentation of their papers reaches the same high level as that of the work they describe. During the review process, a point-by-point response to any review comment(s) should be provided along with the revised manuscript submission.

\section{CONCLUSION}
All papers that are submitted for consideration for an archival status will be reviewed. Papers with archival value (archival papers) require significant contribution to the state of the art based on the final results of research. The scientific contribution should contain sufficient novelty while improving knowledge beyond specific solutions to a particular case study. Any paper that does not address new and innovative aspects of the topics of the meeting may not be included in the final Conference Proceedings. In order to process the reviewing in time, please submit your manuscript via web interface Comet in camera-ready form and in Microsoft Word format before the deadline (available on the web page), in order to be included in the Conference Proceedings.